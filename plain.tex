\documentclass[
tate,
book,
paper=b6j,
twoside,
fontsize=9pt, % 欧文フォントサイズ
jafontsize=9pt, % 和文フォントサイズ
%head_space=17truemm, % 天の余白
%gutter=20truemm, % ノドの余白
%line_length=46zw, % 1行の文字数
%number_of_lines=21, % 行数
%column_gap=10truemm,% 段間
headfoot_verticalposition=5truemm,%本文とヘッダ/フッタの間の空き
]{jlreq}

\usepackage{B6tate-jlreq}% B6tate-jlreq.sty を適用

\begin{document}%
\pagestyle{honbun}% ページスタイル honbun 適用

%タイトルページ
\thispagestyle{empty}
\null\vspace{40.5truemm}% 右余白
\begin{center}
{\LARGE 吾輩は猫である}\hspace{2\zw}{\large 夏目漱石}% タイトル、2文字分空ける、筆者名
\end{center}
\newpage

%以下に本文を書いていく。
\pagestyle{honbun}% ページスタイル honbun に戻す

\section{}




\end{document}
